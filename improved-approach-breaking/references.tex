\section{References}


1. Tam, Jennifer, et al. "Breaking Audio CAPTCHAs." NIPS. 2008. \newline
2. Tam, Jennifer, et al. "Improving audio captchas." Symposium On Usable Privacy and Security (SOUPS). 2008.\newline
3. Aiswarya, K., and K. S. Kuppusamy. "A Study of Audio Captcha and their Limitations."\newline
4. Breaking Google's audio CAPTCHA: http://www.\newline
networkworld.com/article/2278947/lan-wan/breaking-google-s-audio-captcha.html\newline
5. Bursztein, Elie, and Steven Bethard. "Decaptcha: breaking 75\% of eBay audio CAPTCHAs." Proceedings of the 3rd USENIX conference on Offensive technologies. USENIX Association, 2009.\newline
6. Meutzner, Hendrik, et al. "Using automatic speech recognition for attacking acoustic CAPTCHAs: The trade-off between usability and security." Proceedings of the 30th Annual Computer Security Applications Conference. ACM, 2014.\newline
7. Darnstädt, Malte, Hendrik Meutzner, and Dorothea Kolossa. "Reducing the Cost of Breaking Audio CAPTCHAs by Active and Semi-supervised Learning." Machine Learning and Applications (ICMLA), 2014 13th International Conference on. IEEE, 2014.\newline
8. Bursztein, Elie, et al. "The failure of noise-based non-continuous audio captchas." Security and Privacy (SP), 2011 IEEE Symposium on. IEEE, 2011.\newline
9. Sivakorn, Suphannee, Iasonas Polakis, and Angelos D. Keromytis. "The cracked cookie jar: HTTP cookie hijacking and the exposure of private information." Security and Privacy (SP), 2016 IEEE Symposium on. IEEE, 2016.\newline
10. Amodei, Dario, et al. "Deep speech 2: End-to-end speech recognition in english and mandarin." arXiv preprint arXiv:1512.02595 (2015).
