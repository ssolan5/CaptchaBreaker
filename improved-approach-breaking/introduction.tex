\section{Introduction}
\label{sec:intro}
  CAPTCHAs are completely automated tests that tell humans and computer programs apart. Since their inception in 2000, they have been extensively used by websites to prevent spam account creation and messages, but have gained widespread popularity and use in this era of social networking. They have now become a standard security mechanism employed by all major websites. \newline 

CAPTCHAs have long relied on the ability of humans to perform tasks based on visual recognition, cognition and perception easily. So simple text-based or image-based challenges served to distinguish humans from bots. However, a research by Goodfellow et al. at Google recently showed that today's Artificial Intelligence technology can solve even the most difficult variant of distorted text at 99.8\% accuracy. Thus distorted text, on its own, is no longer a dependable test. Sivakorn et al. [9] demonstrated an extremely effective system that solves the state-of-the-art ReCAPTCHA's image based CAPTCHAs with an accuracy of over 70\% and Facebook's image CAPTCHA with an accuracy of over 83\%. \newline

This led us to test the security of audio based CAPTCHA systems by major CAPTCHA providers and websites. Though audio CAPTCHAs were introduced for reasons of accessibility to visually impaired people, the option to listen to an audio challenge instead of solving the standard text or image based challenge exists for normal users too. In other words, they are accessible to bots too. For this reason, we evaluated the security of audio CAPTCHAs from 8 popular CAPTCHA providers and built our own automated solvers using existing speech recognition services. In this paper, we propose to improve the accuracy of our automated solvers using noise reduction techniques. \newline 