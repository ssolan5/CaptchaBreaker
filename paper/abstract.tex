\begin{abstract}

Captchas have become almost ubiquitous, commonly deployed by websites as part of their 
defenses against fraudsters. However visual captchas pose a
considerable obstacle to certain groups of users, such as the visually impaired,
which has necessitated the inclusion of more accessible captcha schemes.
As a result, most captcha services also offer audio challenges as an alternative.

In this paper we conduct a comprehensive exploration of the audio captcha ecosystem,
and present effective low-cost attacks against the audio challenges offered by seven major
captcha services. Motivated by the recent advancements in deep learning, we demonstrate
how off-the-shelf speech recognition services can be misused by fraudsters for trivially
bypassing the most popular audio captchas. Our experimental evaluation highlights the effectiveness 
of our approach as our \system system achieves a success rate of \note{X-Y\%}.
The implications of our study are twofold. First, we find that the wide availability of advanced
speech recognition services has severely lowered the technical capabilities required by fraudsters
for deploying effective attacks, as there is no longer need for custom classifiers. 
Second, we find that the availability of audio captchas poses a significant risk to services,
as we achieve a \note{x\%} higher attack accuracy against \re's audio challenges compared to 
recent state-of-the-art attacks against their image-based challenges. Overall, we argue that it 
is necessary to explore alternative captcha designs that fulfill the accessibility properties 
of audio captchas without undermining the security offered by their visual counterparts.

\end{abstract}
