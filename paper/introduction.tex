\section{Introduction}
\label{sec:intro}

\note{First paragraph sets the context.}
While automated attacks continue to plague the Internet, the number
of users from around the world going online has reached unprecedented levels.
Thus, it is crucial to deploy user-friendly mechanisms for preventing bots without
becoming obstacles to legitimate users.

\note{Second paragraph is more specific. Talk about audio captchas, give some relevant statistics.}

According to the World Health Organization 285 million people have some form of visual impairment~\cite{impaired}.

Captchas present an interesting dilemma from a design standpoint, as they are an exemplary instance 
of usability and security being at direct contradiction. Adding more noise (or distortion) to combat 
automated solvers will also have a significant negative effect on the ability of users to actually
solve challenges.

Previous work has found that popular audio captchas are more difficult than their visual counterparts,
which was also corroborated by a subsequent systematic evaluation of users' ability to solve various types of
captchas~\cite{captchas-are-hard}.


\note{Third paragraph is about what has been done before, what has changed and what is missing.}

\note{Fourth paragraph talks about what we have done, describes our system.}

In this work, we consider multiple leading CAPTCHA providers which also provide an optional audio 
challenge along with the traditional text or image challenge. To solve these challenges, we consider 
multiple online speech recognition APIs (which we will henceforth refer to as ``solvers'').

\note{Fifth paragraph gives some of our important results.}

Overall, the main contributions of our work are:

\begin{itemize}

\item We present an extensive evaluation of the audio captcha ecosystem, and demonstrate how available services 
can be broken using off-the-shelf speech recognition services.

\item We bla bla bla.

\item We bla bla bla.

\end{itemize}
