\section{Introduction}
\label{sec:intro}

While automated attacks continue to plague the Internet,
the number of of online users around the world has reached unprecedented levels.
Thus, it is crucial to deploy user-friendly mechanisms for preventing bots without
becoming obstacles to legitimate users. Traditionally captchas comprise the first 
line of defense, and attempt to differentiate between actual users and bots by 
presenting a challenge that requires some form of task that is considered hard 
for a computer~\cite{captcha}. While visual captchas (i.e., text and image based challenges) 
have evolved significantly since their initial incarnation, with Google's recent \re system 
attempting to bridge the gap between security and usability, audio captchas have remained 
relatively unchanged.

Audio captchas are deployed as an alternative type of challenge, targeting
visually-impaired users that would otherwise not be able to pass the presented
challenge. According to the World Health Organization 285 million people have 
some form of visual impairment~\cite{impaired}, rendering audio captchas a
crucial component of an inclusive world wide web. While audio captchas may still present
an obstacle to certain users~\cite{sauer2008towards,bigham2008inspiring,bigham2009evaluating},
they are currently the de facto alternative offered by captcha services.

While audio captchas have not received as much attention from the security community as visual captchas,
previous work has presented automated attacks against audio captchas~\cite{Sano2013,Bursztein2009,
meutzner2014using,tam2009breaking,bursztein2011failure}. These studies have demonstrated attacks
that achieve high accuracy by building customs classifiers using machine learning techniques.
However, the tremendous advancements of the past years in deep learning technology, have
resulted in significant improvements in speech recognition systems, recently achieving
results that approach human performance~\cite{ibm_blog,saon2017english}. Motivated
by these advancements, and the recent work by Sivakorn et al. against visual
captchas~\cite{sivakorn:eurosp16}, we investigate how speech recognition services
can be misused for breaking existing audio captcha systems. By exploring how effective
widely-available speech recognition systems are at transcribing audio challenges despite 
the presence of noise, typically present in captcha challenges, we can assess the robustness 
of the captcha ecosystem.

In this work we present \system, a modular system that leverages existing speech recognition services
for solving audio captcha challenges. Specifically, we leverage the
APIs offered by IBM's Watson, Facebook's Wit, and Google's Cloud Speech. In a nutshell, our system works as
follows. First, our browser automation module visits a page protected by one of the captcha services,
and obtains the audio challenge. Next, after minimal preprocessing, the audio recording is submitted 
to one of the speech recognition services. Upon receiving the transcription from the service, we 
process the text based on the properties of the specific captcha service (e.g., transcribing to alphanumeric 
from NATO's phonetic alphabet) and prepare the response to the challenge. This is then handled by
the browser automation module that inputs the response and checks whether it was accepted by the captcha
service.

We evaluate \system again \no captcha systems, including popular services like Google \re,
Apple's and Microsoft's captchas. Our experiments demonstrate the effectiveness of our approach,
as we are able to (greatly) surpass the threshold commonly set in prior work for considering
a captcha system broken, with the highest accuracy achieved against \re -- the most prevalent 
captcha system.

Surprising findings: every single one broken. Accent matters. Higher accuracy than prior work. Much higher accuracy than
visual captchas. Old \re harder than new \re. 
Apart from facilitating visually-impaired users, audio captchas introduce an attack
surface that presents a significantly smaller obstacle to attackers compared to visual captchas.

\note{Important takeaways. Broader impact.}

Overall, the main contributions of our work are:

\begin{itemize}

\item We present the first, to our knowledge, comprehensive evaluation of 
the audio captcha ecosystem against low-cost 
%by demonstrating how available   services 
%can be broken using 
attacks using off-the-shelf speech recognition services.

\item We bla bla bla.

\item We bla bla bla.

\end{itemize}
