\section{Discussion}
\label{sec:discussion}

While text-based captchas have been the most common type of challenge for years, they have been rendered
obsolete by the demonstration of generic solvers that do not require specific training per captcha system~\cite{185128}.
Recently, researchers demonstrated how deep learning neural networks can be leveraged for breaking image-based 
semantic captchas~\cite{sivakorn:eurosp16}. Our work highlights the ineffectiveness of audio captchas, as 
existing speech recognition systems can be readily applied to transcribing the audio challenges.
As such, it is crucial to design alternative methods for preventing automated attacks; while researchers 
have also considered cognitive games as promising alternatives, they have been found to be vulnerable to 
attacks~\cite{mohamed2017security}.

While prior work has typically explored how additional noise can prohibit automated attacks, these are 
short-lived approaches as machine learning technologies continue to improve and evolve at a radical rate.
Furthemore, audio captchas present an interesting dilemma from a design standpoint, as they are an considerable 
constrained in regards to the ``security versus usability'' tradeoff. Generally, adding more noise (or distortion) 
to combat automated solvers can have a severely negative effect on the ability of users to actually solve challenges.
In the case of audio captchas, this constraint is more pronounced, as human hearing is more error prone than 
vision~\cite{o2009auditory,shinn2008object}.

When taking into consideration that prior work has found popular audio captchas to be much more difficult 
than their visual counterparts~\cite{bigham2009evaluating}, which was also corroborated by a subsequent 
systematic evaluation of users' ability to solve various types of captchas~\cite{captchas-are-hard}, further
distorting existing audio captchas would further hinder visually impaired users from accessing the Web.
Thus, while designing a robust captcha scheme remains an open problem, it is important 
to explore designs that do not present obstacles to certain groups of users (e.g., the visually impaired). 
While this may prove to be an insurmountable constraint, we argue that the current approach of offering audio
challenges is \note{misguided} as they remain an obstacle to visually impaired users while offering 
adversaries a pathway to more accurate attacks.
