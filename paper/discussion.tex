\section{Discussion}
\label{sec:discussion}

Even though text-based captchas have been prevalent for years they should be considered
obsolete, as prior work has demonstrated generic solvers that do not require specific training per captcha system~\cite{185128}.
Moreover, researchers recently demonstrated how deep learning neural networks can be leveraged for breaking image-based 
semantic captchas~\cite{sivakorn:eurosp16}. In a similar vain, our work highlights the ineffectiveness of audio 
captchas, as existing speech recognition systems can be readily misused for transcribing the audio challenges.
As such, it is crucial to design alternative methods for preventing automated attacks. While researchers 
have also considered cognitive games as promising alternatives, they have been found to be vulnerable to 
attacks~\cite{mohamed2017security}.

One could argue that the inclusion of additional noise can prohibit automated attacks. However, we believe this to be
a short-lived approach as machine learning technologies continue to improve and evolve at a radical rate and may soon
match or even surpass human performance at such tasks.
Furthermore, audio captchas face an important inherent limitation from a design standpoint, as they are particularly
constrained by the ``security versus usability'' tradeoff. Generally, adding more noise (or distortion) 
to combat automated solvers can have a severely negative effect on the ability of users to actually solve challenges.
In the case of audio captchas specifically, this constraint is more pronounced, as human hearing is more error prone 
than vision~\cite{o2009auditory,shinn2008object}. Furthermore, the serialized nature of the auditory signal limits 
one's ability to focus on a specific segment, as opposed to a visual challenge where a user can apply more effort and 
time focusing on a specific segment of the challenge.

When taking into consideration that prior work has found popular audio captchas to be much more difficult 
than their visual counterparts~\cite{bigham2009evaluating,captchas-are-hard},
%, also corroborated by a subsequent 
%systematic evaluation of users' ability to solve various types of captchas~\cite{captchas-are-hard}, 
adding additional distortion to existing audio captchas would further hinder visually impaired users from accessing the Web.
Thus, even though designing a robust captcha scheme remains an open problem, it is important 
to explore designs that do not present obstacles to certain groups of users (e.g., the visually impaired).
While this may prove to be an insurmountable constraint, we argue that the current approach of offering audio
challenges is a half-measure, as they remain an obstacle to visually impaired users while offering 
adversaries a pathway to more accurate attacks.
