\section{Related Work}
\label{sec:related}

\cite{sivakorn:eurosp16} broke Google's \re.

Sano~\cite{Sano2013} broke audio \re 2013 (52\%).
Burzstein and Bethard broke Ebay captcha~\cite{Bursztein2009}.

Tam et al.~\cite{tam2008improving} were the first to evaluate the robustness of audio captchas against automated attacks.

``Sauer et al. found that
six blind participants had a success rate of only 46\% in
solving the audio version of the popular reCAPTCHA''~\cite{sauer2008towards}.

Meutzner et al.~\cite{meutzner2014using} broke \re 2014.

According to Shirali-Shahreza et al. \cite{shirali2011accessibility} three groups of people
have trouble with visual challenges - Visually impaired, who constitute 2.6\% of the world's
population, users with dyslexia, and users suffering from motor impairment diseases like Parkinson's.

Chellapilla et al. \cite{Chellapilla} on designing Human-friendly Interaction Proofs (HIPs)
argue that HIPs must approach a success rate of at least 90\%. Adding to this, they also state that the
attack on such a system by automated computer software/scripts must not be successful for more than 1 in
10,000 challenges (0.01\%). A study conducted by Sauer et al. \cite{sauer2008towards} which dealt with
evaluating audio CAPTCHAs with visually impaired users found that they could only solve the challenges
at a rate of 46\%, 44\% less then what is supposed to be. They also found that the average amount of time
taken to correctly solve an audio challenge was 65.64 seconds, which is greater than the 51 seconds that
is suggested as the time to complete a CAPTCHA \cite{schluessler2007bot}.

While conducting a study with blind high school students to inspire them to pursue Computer Science, Bigham et al.
found that when the students were presented with an audio CAPTCHA, none of them were able to solve the challenge and
their sighted instructors ended up solving the visual version instead \cite{bigham2008inspiring}. In a subsequent
study conducted by Bigham et al with 89 blind users, they found that the users achieved only a 43\% success rate in
solving 10 popular audio CAPTCHAs \cite{bigham2009evaluating}. 

In the same study \cite{bigham2009evaluating}, it was also found that screen readers used by blind users speak over
playing CAPTCHAs. As users navigate to the answer box, the accessibility software continue speaking the interface while
talking over the playing audio challenge. A playing audio challenge does not pause for solvers as they type their answer
and reviewing an audio CAPTCHA is cumbersome, often requiring the user to start again from the beginning. Also, replaying
an audio CAPTCHA requires solvers to navigate away from the answer box in order to access the controls of the audio player.
Thus, the authors proposed a system and optimized the interface of popular audio CAPTCHA services without altering the
underlying implementation and found that the performance increased to 59\%. 
