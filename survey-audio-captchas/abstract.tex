\section{Abstract}\mbox{}\
\label{sec:abstract}

CAPTCHA is a backronym for Completely Automated Public Turing test to tell Computers and Humans Apart. As the name suggests, they are automated tests designed to control access to computer systems by distinguishing humans from bots. Since their launch in 2000, websites have used them extensively to prevent bot account creation and spam. There are many types of tests devised for CAPTCHAS in the past. The most common CAPTCHAS are text based, where an image is shown to the users which contains distorted text. Users are asked to read and recognize this text and type the text they see into a textbox that is situated next to the image. Some other common types of CAPTCHAs include tasks of selecting a single or multiple similar images among many heterogeneous images, solving math puzzles, recognizing 3D text, clicking or dragging moving elements, locating text or numbers in a video, etc. \newline

For a long time now, researchers have tested the strength of text CAPTCHAs and using the latest advancements in computer vision, have been able to break text CAPTCHAs with an accuracy of up to 77\% \cite{ebay:machine-noise}. The version 2 ReCAPTCHA by Google employs checkbox and image CAPTCHAs which again, after scrutiny, have been rendered broken with an accuracy of 70.78\% \cite{recurrent:machine-noise}. Ever since CAPTCHAs have been introduced, companies have built their own CAPTCHA systems and have used them to prevent bot activity on their web pages. Also, a number of enterprise solutions that offer CAPTCHA systems as a service have since emerged. Although CAPTCHAs have become very common, users often encounter CAPTCHAs which are hard even for humans to solve. As there is very less or no margin for error to prevent bot activity, users often end up making multiple attempts to solve a challenge. It has been shown that CAPTCHA challenges frustrate users and it negatively impacts the experience and traffic of a website \cite{distil}.\newline

Commercial CAPTCHA services can be utilized by users by signing up with one of the CAPTCHA providers and including a few lines of their code on the target web pages. ReCAPTCHA is one such service offered by Google. Text and image challenges are the most common CAPTCHAs offered by these services which are implemented on web pages to prevent bots from filling web forms automatically. But a drawback with these CAPTCHA systems is that certain groups of people, especially the visually impaired, find it difficult to use them. Thus, audio challenges were introduced specifically to help computer systems be accessible to the visually impaired. The caveat with this approach is that these audio CAPTCHAs open a new ground for exploitation for attackers.\newline

While there has been a lot of prior work done on breaking text and image CAPTCHAs, audio CAPTCHAs have not been studied to such great detail. In this work, we present a survey of all the audio CAPTCHA services that are currently used and show how we used our system to automate the solving of these challenges using existing, off-the-shelf speech recognition services. We found that the latest version of ReCAPTCHA is very vulnerable to our attack.\newline

  


